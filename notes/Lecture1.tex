\documentclass{article}
\usepackage[utf8]{inputenc}    % For UTF-8 character encoding
\usepackage[T1]{fontenc}       % For proper font encoding
\usepackage{lmodern}           % Improved font rendering
\usepackage{amsmath}   % For advanced mathematical formatting
\usepackage{amssymb}   % For mathematical symbols
\usepackage{geometry}  % Adjust page margins
\usepackage{enumerate} % For custom lists
\usepackage{xcolor}  % for coloring
\usepackage{amsthm}
\usepackage{pdfpages}
\newtheorem{theorem}{Theorem}[section]
\newtheorem{lemma}[theorem]{Lemma}
\newtheorem{corollary}[theorem]{Corollary}
\newtheorem{definition}[theorem]{Definition}
\usepackage{listings}  % for code listings

\lstset{frame=tb,
  language=C,
  aboveskip=3mm,
  belowskip=3mm,
  showstringspaces=false,   
  columns=flexible,
  basicstyle={\small\ttfamily},
  numbers=none,
  numberstyle=\tiny\color{gray},
  keywordstyle=\color{blue},
  commentstyle=\color{brown},
  stringstyle=\color{orange},
  breaklines=true,
  breakatwhitespace=true,
  tabsize=3
}
\geometry{top=1in, bottom=1in, left=1in, right=1in}

\begin{document}

\title{Lecture 1: Introduction}
\author{Wang Xiyu}
\date{}
\maketitle
\section{Overview}
\begin{itemize}
    \item Week 1-3: Classical AI, search algorithms
        \begin{enumerate} 
            \item Uninformed search
            \item Local search: hill climbing
            \item Informaed search: A$^*$
            \item Adversarial search Minimax
        \end{enumerate}
    \item Week 4-7: Classical ML
        \begin{enumerate}
            \item Decision trees 
            \item Linear/Logistic regression 
            \item Kernels and support vector machines
            \item "Classical" unsuperivese learning
        \end{enumerate}
    \item Week 10-12: Modern ML
        \begin{enumerate}
            \item Neural networks
            \item Deep learning 
            \item Sequential data
        \end{enumerate}
    \item Week 13: Misc.
\end{itemize}

\section{AI: Computers Trying to Behave Like Humans}

\begin{itemize}
    \item \textbf{PEAS Framework:}
    \begin{itemize}
        \item \textbf{Performance measure:} define “goodness” of a solution
        \item \textbf{Environment:} define what the agent can and cannot do
        \item \textbf{Actuators:} outputs
        \item \textbf{Sensors:} inputs
    \end{itemize}
    
    \item Agent function is sufficient.
    
    \item Common agent structures (to define an AI agent):
    \begin{itemize}
        \item Reflex
        \item Goal-based
        \item Utility-based
        \item Learning
        \item (Others possible; can mix and match!)
    \end{itemize}
    
    \item Exploration vs exploitation
\end{itemize}

























/
\end{document}
