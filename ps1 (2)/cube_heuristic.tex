\documentclass{article}
\usepackage[utf8]{inputenc}    % For UTF-8 character encoding
\usepackage[T1]{fontenc}       % For proper font encoding
\usepackage{lmodern}           % Improved font rendering
\usepackage{amsmath}   % For advanced mathematical formatting
\usepackage{amssymb}   % For mathematical symbols
\usepackage{geometry}  % Adjust page margins
\usepackage{enumerate} % For custom lists
\usepackage{xcolor}  % for coloring
\usepackage{amsthm}
\usepackage{pdfpages}
\newtheorem{theorem}{Theorem}[section]
\newtheorem{lemma}[theorem]{Lemma}
\newtheorem{corollary}[theorem]{Corollary}
\newtheorem{definition}[theorem]{Definition}
\usepackage{listings}  % for code listings

\lstset{frame=tb,
  language=C,
  aboveskip=3mm,
  belowskip=3mm,
  showstringspaces=false,   
  columns=flexible,
  basicstyle={\small\ttfamily},
  numbers=none,
  numberstyle=\tiny\color{gray},
  keywordstyle=\color{blue},
  commentstyle=\color{brown},
  stringstyle=\color{orange},
  breaklines=true,
  breakatwhitespace=true,
  tabsize=3
}
\geometry{top=1in, bottom=1in, left=1in, right=1in}

\begin{document}

\title{Consistency and Admissibility Proof for Total Manhattan Distance Heuristic for Cube problem}
\author{Wang Xiyu}
\date{}
\maketitle
Simple Manhattan distance  heuristic where onlt horizontal and vertical moves are allowed is trivial; 

each move costs 1, and each horizontal move will change the horizontal index by 1, so do vertical moves 
1 2 3 4
4 1 2 3
2 3 4 1 

the largest possible manhatten diatance between a pair is $\lceil\frac{r}{2}\rceil + \lceil\frac{c}{2}\rceil$
, therefore optimistically the max cost of one swap will be $\lceil\frac{r}{2}\rceil + \lceil\frac{c}{2}\rceil$. 
A generous upper bound to the total manhatten distance from the goal state will be 
$\frac{r \times c}{2}(\lceil\frac{r}{2}\rceil + \lceil\frac{c}{2}\rceil)$ 




\end{document}
